\section{Build And Test} \label{buildAndTest}

Per concludere, è possibile reperire sul: \href{https://github.com/Alfystar/SOA-prj}{Project Git Repository}, un recup
delle features implementate e una lieve guida alla compilazione e caricamento della libreria nel Kernel.

I comandi necessari sono stati embeddati all'interno del file \verb|Makefile|.

Il programma è stato scritto usando come editor Visual Studio Code, con l'enviroment impostato per leggere i sorgenti
del Kernel ``Linkati'' tramite link simbolico alla directory \verb|/usr/src| in questa maniera::

\begin{verbatim}
/usr/src
|-- linux -> linux-headers-5.11.0-16
|-- linux-generic -> linux-headers-5.11.0-16-generic
|-- linux-headers-5.11.0-16
|-- linux-headers-5.11.0-16-generic
\end{verbatim}

Realizzabile con i permessi di root usando da terminale \verb|ls -s <Target dir> <Link name>|.
