\section{SysCall Discovery} \label{SysCallDiscovery}

Il Sottosistema che si occupa di ricercare la posizione all'interno della memoria della \verb|sys_call_table| è stato
sviluppato partendo dal lavoro del professore Francesco Quaglia
(\href{https://github.com/FrancescoQuaglia/Linux-sys_call_table-discoverer}{Git repository}).

In particolare il progetto è stato ``forkato'' al commit:
\begin{verb}6eafea1bcacb0ee7f81f24657ce03290d98c7947\end{verb}; esso è stato modificato e
adattato per diventare una libreria del sistema che permette di esporre le funzionalità di modifica e ripristino
delle syscall ``Libere'' (le syscall che puntano a \verb|sys_ni_syscall|) necessarie.

Esso comincia una ricerca lineare lungo i primi 4GB della Ram, e verifica, byte per byte, la struttura della memoria
sottostante, se essa coincide con quella della \verb|sys_call_table| nell'area di memoria in esame.
Se il test ha successo, allora l'area di memoria trovata viene considerata la \verb|sys_call_table|.
Il rischio di falsi positivi tuttavia esiste e per minimizzare un simile rischio sono stati utilizzati gli
spiazzamenti di 7 diverse SysCall non implementate, così da rendere minimo il rischio di falsi positivi.

\subsection{Enhanced Features} \label{SysCallDiscovery_EnhancedFeatures}
In aggiunta alle funzionalità offerte dal progetto base del professore, il Sottosistema è in grado di effettuare il
restore delle entry modificate dal sistema durante lo smontamento del modulo.
Ciò è verificabile montando e smontando ripetutamene il modulo: se le syscall non venissero ripristinate sarebbe
impossibile individuare nuovamente la tabella.
\newpage


\section{Tag-Based Data Exange} \label{tbde}

Il cuore del progetto (lo scambio di messaggi tra processi) viene realizzato dal TBDE SubSystem.

Per poter svolgere la sua funzione, si è scelto di usare come strutture dati:
\begin{description}
\item[AVL-Tree] per indicizzare le stanze per \tagSys e \keySys:\\
    Il sistema usa 2 Alberi-AVL di ricerca, \tagTree e \keyTree, per indicizzare in maniera efficiente le
    \textbf{Stanze} in base alla chiave di ricerca usata (\tagSys e \keySys). \\
    L'albero principale è il \tagTree, mentre il \keyTree è di supporto per la ricerca efficiente delle
    stanze con visibilità ``pubblica''.\\
    Gli alberi sono sincronizzati mediante degli \texttt{Spinlock RW} che permettono di svolgere in \textbf{reale
    parallelismo tutte le operazioni comunicazione} e vincolano a 1 Thread alla volta per le operazioni di:
    \begin{itemize}
    \item \textit{Aggiunta} \\
        Creazione di una stanza, con visibilità pubblica o privata, usando permessi richiesti:\\
        (\texttt{\underline{Open}(aperta a tutti)}/\texttt{\underline{Private}(solo ai Thread del processo creante)})
    \item \textit{Rimozione} \\
        Rimozione di una stanza, se i permessi lo permettono, il SUPER-UTENTE è sempre considerato valido
    \item \textit{Status-Print} \\
        Operazione eseguita nel Char-Device, per generare lo ``Screenshot'' dello stato del sistema in quell'istante
        temporale. Per garantire la coerenza dei dati letti nel trovare in quale stanze vi sono dei \Reader in attesa e
        quanti, viene fatto un Lock generale del sistema al livello degli Alberi-AVL, l'operazione blocca
        temporaneamente il procedere del sistema ma permette di eseguire uno ``Screenshot'' coerente.
        Il lock non è problematico poiché è un operazione assolutamente infrequente e User-driven.
\end{itemize}

\begin{footnotesize}
Per maggiori informazioni sulla scelta di usare \texttt{Alberi-AVL RWLock} rispetto a una lista
RCU, fare riferimento all'approfondimento presente nella sezione \nameref{treePerf}.
\end{footnotesize}

I nodi degli Alberi puntano alle \textbf{Stanze} allocate nel sistema, non una copia personale per l'albero, ma
proprio la stessa istanza, ciò implica che i 2 alberi devono essere consistenti tra di loro, e che il lock
descritto precedentemente, copra contemporaneamente entrambe le strutture dati.

\item[Stanze] (\textit{Room}):\\
Il nodo della \textbf{Stanza} è da considerare come un oggetto allocato nell'heap del Kernel, si è reso
quindi necessario usare le API dei \texttt{refcount} presenti all'interno del Kernel. Il contatore cresce all'aumentare
degli oggetti che puntano l'oggetto ``Stanza'' e decresce alla loro rimozione, il Thread che porta
effettivamente il contatore della ``Stanza'' a 0 esegue anche le operazioni di liberazione della memoria.

\item[\Topic] (\exangeRoom):\\
Ogni stanza ha a sua volta 32 \exangeRoom (i \Topic), che permettono lo scambio effettivo delle informazioni
tra 1 \Writer e gli n-\Reader precedentemente in attesa su questo \Topic dall'arrivo del precedente \Writer.

Ognuna delle \exangeRoom è anch'essa un oggetto del sistema, e viene gestita con le API dei \texttt{refcount},
l'unica differenza è che per le \exangeRoom risulta ammissibile che la stanza sia valida con un contatore a 0,
questo perché le \exangeRoom contengono i \Reader in attesa su quel \Topic, ed è possibile che un \Topic sia
senza \Reader in attesa.

Questa scelta ha però generato la necessità di un uso non proprio lecito delle API di \texttt{refcount},
poiché, per queste API, un \textit{refcount = 0} equivale a un area di memoria che deve essere liberata, e non viene
permesso l'incremento. Per ovviare al problema , si è sostituita la \verb|refcount_inc| con il suo equivalente
\verb|atomic_inc|, che però bypassa i check di consistenza.
\end{description}

\begin{scriptsize}
Per approfondire il comportamento del sistema nello scambio e come queste 3 strutture dati interagiscono per permettere
lo scambio di messaggi, fare riferimento alla sezione \nameref{exangeDataProtocol}.
\end{scriptsize}

\newpage

\subsection{Exange Data protocol} \label{exangeDataProtocol}

Fin dall'inizio della sua progettazione, l'obiettivo principale del sistema è stato permettere a un \Writer di
comunicare con tutti gli n-\Reader precedente accodati sul \Topic, nel minor tempo possibile ed evitando lock
di sincronizzazione, il risultato è un sistema rapido e snello, che rallenta solo durante la Creazione o Rimozione
delle stanza (operazioni supposte infrequenti) e nelle operazioni di comunicazione richiede un unico \textit{Soft Lock}
implementato come un \textit{custom lock di enable/disable Free} (vedi \nameref{memFreeLockProt}), che comunque
rallenta al più per un paio di cicli macchina, ma evita una rara, seppur possibile, corsa critica nell'accesso a un area
di memoria.

Per accodare e addormentare i \Reader in attesa sono state utilizzate le  \texttt{API di Wait-Queue} del Kernel che già
evitano il problema del \underline{Wake up Lost Problem}.

Il protocollo di scambio dati è progettato per garantire che un \Writer possa prendere tutti i \Reader accodati fino al
momento del suo arrivo, evitando che 2 \Writer possano parlare agli stessi \Reader.

Esso consiste nel lasciare sempre disponibile ai \Reader del sistema, una coda su cui andare a dormire, questa coda
viene \textbf{Atomicamente Swappata}  con una vuota all'arrivo di un \Writer.

Ciò fa ``catturare'' tutti i \Reader serializzati fino al suo arrivo e svegliare i \Reader addormentati senza
concorrenza con altri \Writer, che possono eseguire in tranquillità parlando con i nuovi \Reader in attesa.


\subsubsection{Reader e Writer Protocol}
Vediamo ora lo pseudo-codice dei 2 flussi:

\begin{algorithm}
\caption{\Writer ExangeDataProtocol}\label{writeExange}
\begin{algorithmic}[1]
\Procedure{tagSend}{$tag, level, msg$}
\State {\color{orange}$ReadLock(\tagTree)$}
\State $roomSearch \gets TreeSearch(tag, \tagTree)$
\If {$roomSearch == FOUND$}
    \State $RoomRefInc(Room)$
    \State {\color{orange}$ReadUnLock(\tagTree)$}
    \If {$permissionCheck(roomSearch) == True $}
        \State $newER \gets makeExangeRoom()$ \Comment{La coda vuota da Sostituire}
        \State $exange \gets CompareSwap(newER, currER(level))$\Comment{Swap atomico}
        \State $ExangeRefInc(exange)$
        \State $copyToExangeRoom(msg)$
        \State $wakeUp(exange.WaitQueue)$
        \State {\color{blue} $tryFreeExangeRoom(exange)$}
        \State $TryFreeRoom(roomSearch)$
    \Else
        \State $TryFreeRoom(roomSearch)$
        \State \textbf{return} $-1$\Comment{Operation fail}
    \EndIf
\Else
    \State {\color{orange}$ReadUnLock(\tagTree)$}
    \State \textbf{return} $-1$\Comment{Operation fail}

\EndIf
\EndProcedure
\end{algorithmic}
\end{algorithm}

\newpage

\begin{algorithm}
\caption{\Reader ExangeDataProtocol}\label{readExange}
\begin{algorithmic}[1]
\Procedure{tagRecive}{$tag, level, msg$}
\State {\color{orange}$ReadLock(\tagTree)$}
\State $roomSearch \gets TreeSearch(tag, \tagTree)$
\If {$roomSearch == FOUND$}
    \State $RoomRefInc(Room)$
    \State {\color{orange}$ReadUnLock(\tagTree)$}
    \If {$permissionCheck(roomSearch) == True $}
        \State {\color{red}$exange \gets currER(level)$\Comment{Ottengo la stanza attualmente sincronizzata}}
        \State {\color{red}$ExangeRefInc(exange)$}
        \State $InterruptableSleep(exange.WaitQueue)$
        \State ...
        \State [Wake Up Event]
            \If {Wake up for signal}
                \State {\color{blue} $ExitFreeExangeRoom(exange)$}
                \State \textbf{return} $-1$\Comment{Signal Wake up}
            \EndIf
        \State $copyFromExangeRoom(msg)$ \Comment{Normal Wake up}
        \State {\color{blue} $tryFreeExangeRoom(exange)$}
        \State $TryFreeRoom(roomSearch)$
    \Else
        \State $TryFreeRoom(roomSearch)$
        \State \textbf{return} $-1$\Comment{Operation fail}
    \EndIf
\Else
    \State {\color{orange}$ReadUnLock(\tagTree)$}
    \State \textbf{return} $-1$\Comment{Operation fail}
\EndIf
\EndProcedure
\end{algorithmic}
\end{algorithm}

\textbf{N.B. Il protocollo descritto permette di far eseguire i \Writer e \Reader dello scambio dei messaggi in reale
parallelismo!!!}

Il protocollo risulta essere fluido e non sono presenti lock espliciti all'interno della \exangeRoom all'infuori
dell'attesa del \Reader, necessaria e desiderata.

Da osservare i comandi in {\color{blue} blu} del codice, essi evidenziano i 2 diversi tipi di FreeExangeRoom:
\begin{description}
\item[Exit Free]:\\
In caso il contatore \textbf{scenda a 0}, non elimina la stanza, poiché l'area di memoria è in realtà ancora valida e
raggiungibile
\item[Try Free]:\\
Quando il contatore \textbf{scende a 0}, si può effettivamente procedere all'eliminazione dell'oggetto, poichè non è più
riferito da nessuno.
\end{description}

\rule{\textwidth}{1pt}
\subsubsection{Memory Free lock Protocol}\label{memFreeLockProt}

Purtroppo la sezione {\color{red} Rossa} del \nameref{readExange} necessita di poter essere
eseguita \textbf{Atomicamente} per garantire che l'incremento del contatore avvenga prima che lo stesso possa essere
eliminato da una {\color{blue}$TryFreeRoom(roomSearch)$}. Per eliminare la corsa critica nell'acquisizione
della \textit{exangeRoom} e il successivo incremento del \textit{refcount}, si è scelto di implementare
un protocollo di \texttt{\textbf{Memory Free lock}}:

Esso è inspirato alla logica delle liste RCU, con una differenza nel modo di attendere il \textbf{tempo di grazia}:
classicamente è il \Writer a dover attendere il periodo di grazia, in questa versione invece è semplicemente il Thread
più ``lento`` ad eseguire il lavoro (dovuto magari a rallentamenti per lo scheduler o la priorità del Thread).

Per risolvere la corsa critica è stata scritto un piccolo Spinlock custom (reperibile in tbde.h):


\begin{indentPar}{2.5cm}
\begin{small}
Custom lock define:
{\color{red}\begin{verbatim}
#define freeMem_Lock(atomic_freeLockCount_ptr)              \
  do {                                                      \
    preempt_disable();                                      \
    atomic_inc(atomic_freeLockCount_ptr);                   \
  } while (0)

#define freeMem_unLock(atomic_freeLockCount_ptr)            \
  do {                                                      \
    atomic_dec(atomic_freeLockCount_ptr);                   \
    preempt_enable();                                       \
  } while (0)
\end{verbatim}}

{\color{blue}\begin{verbatim}
#define waitUntil_unlock(atomic_lockCount_ptr)              \
  do {                                                      \
    preempt_disable();                                      \
    while (arch_atomic_read(atomic_lockCount_ptr) != 0) {   \
    };                                                      \
    preempt_enable();                                       \
  } while (0)
\end{verbatim}}
\end{small}
\end{indentPar}


Queste API sono usate:
\begin{itemize}
\item
Dai \Reader (\verb|freeMem_Lock| e \verb|freeMem_unLock|) attorno alla sezione critica ({\color{red}Rossa})

\item
All'interno della {\color{blue}$TryFreeRoom(roomSearch)$} (la \verb|waitUntil_unlock|) per poter verificare,
\textbf{solo al raggiungimento dello 0}, che non sia in corso nessuna operazione rischiosa in memoria, e in tal caso
attendere la fine della sezione critica e reagire di conseguenza alle modifiche del sistema.
\end{itemize}

Per minimizzare al massimo l'impatto di questo \texttt{Soft-lock}, ogni \Topic è dotato del suo personale memoryLock,
così da circoscrivere ai soli Thread interessanti il rallentamento. Va in oltre precisato che il rallentamento è
sperimentato solo quando il contatore arriva a 0, durante i valori superiori del refcount, il contatore è gestito
atomicamente dalle API del \textit{refcount}, senza l'uso di sezioni critiche.

% \newpage

\subsection{AVL-Tree Performance} \label{treePerf}
Dopo aver visto l'architettura del sistema, ragionevolmente ci si può chiedere:

\begin{center}
\textit{Perchè usare degli Alberi-AVL con dei RWlock invece di una lista RCU?}
\end{center}

Per rispondere a questa domanda, e per capire il perchè siano state fatte le scelte realizzative è opportuno tenere in
mente le priorità del sistema:

\begin{enumerate}
 \item Scambio di messaggi rapido e affidabile tra i processi.
 \item Parallelizzare le operazioni (in particolare quelle più frequenti)
 \item Evitare operazioni inutili
\end{enumerate}

\subsubsection{RWLock AVL-Tree vs RCU List} \label{AVL_Vs_RCUList}

Andiamo ora a comparare le prestazioni possibili tra:

\begin{minipage}{0.45\linewidth}
\begin{center}
\texttt{Albero-AVL con Spinlock RW}
\end{center}

\begin{itemize}
 \item[\faSmileO] \Writer e \Reader\\
possono accedere in contemporanea e ci mettono al {\color{red}massimo O($log(N)$)} per arrivare alla stanza di loro
interesse
\item[\faMehO\faFrownO\faSmileO] Inserire/Rimuovere\\
nodi con {\color{orange}1 Thread}, dovendo fermare i \Writer e \Reader al più per {\color{red}O($log(N)$)}

\end{itemize}
\end{minipage}
\hfill
\begin{minipage}{0.45\linewidth}
\begin{center}
\texttt{Lista RCU}
\end{center}

\begin{itemize}
 \item[\faFrownO] \Writer e \Reader\\
possono accedere in contemporanea e ci mettono al {\color{red}massimo O($N$)} per arrivare alla stanza di loro
interesse
 \item[\faMehO\faSmileO\faFrownO] Inserire/Rimuovere\\
nodi con {\color{orange}1 Thread}, senza mai fermare i \Writer e \Reader, ogni operazione richiede
{\color{red}O($N$)}
\end{itemize}
\end{minipage}

Essendo entrambe le strutture dati organizzate come dei nodi da accedere, non è possibile avere su nessuna delle 2 un
accelerazione HW particolare, e specie per sistemi molto grandi, \textbf{l'Albero AVL scala meglio lungo un periodo},
per capire meglio l'ultima affermazione consideriamo il seguente esempio al caso peggiore:

\rule{\linewidth}{0.5pt}
Nel sistema sono presenti {\color{teal}256 stanze}, il SO ha a disposizione {\color{teal}8 Thread} per
poter eseguire le operazioni, prendiamo come periodo {\color{teal}256 operazioni} (supponiamo che ogni
mossa tra un nodo e l'altro richieda 1 operazione), ecco le prestazioni massime che ci possiamo aspettare, supponendo
che tutte le operazioni puntino al 256-esimo nodo:

\begin{minipage}{0.45\linewidth}
\begin{center}
\texttt{Albero-AVL con Spinlock RW}
\end{center}
\begin{itemize}
 \item[\faSmileO] Search \\
 $8 [Thread] \cdot \frac{256 [OpTime]}{log_2(256 [Room])} = 256 [Op]$

 \item[\faSmileO] Insert/Delete (alternate) \\
 $1 [Thread] \cdot \frac{256 [OpTime]}{log_2(256 [Room])} = 32 [Op]$
\end{itemize}
\end{minipage}
\hfill
\begin{minipage}{0.45\linewidth}
\begin{center}
\texttt{Lista RCU}
\end{center}
\begin{itemize}
 \item[\faFrownO] Search\\
  $8 [Thread] \cdot \frac{256 [OpTime]}{256 [Room]} = 8 [Op]$
 \item[\faFrownO] Insert/Delete (alternate) \\
  $1 [Thread] \cdot \frac{256 [OpTime]}{256 [Room]} = 1 [Op]$
\end{itemize}
\end{minipage}

\rule{\linewidth}{0.5pt}

Inoltre, all'aumentare dei nodi e del periodo la vittoria dell'Albero-AVL risulta particolarmente evidente, queste
considerazioni hanno portato all'esclusione delle liste RCU per questa specifica implementazione del sistema, specie
considerando che è possibile modificare a \RunTime il numero massimo di Stanze, facendo aumentare sempre più il
vantaggio di usare un Albero-AVL.

\subsubsection{RWLock AVL-Tree vs RCU AVL-Tree} \label{AVL_Vs_RCUAvl}

Alla fine di questi calcoli viene normale chiedersi:

\begin{center}
\textit{Perché non usare direttamente una libreria di Alberi-AVL RCU?}
\end{center}

Qui la risposta è da trovare nei paper pubblicati: ad oggi (\today) l'argomento è dibattuto in ambito universitario, e
sono state proposte delle soluzioni per estendere la logica RCU anche agli Alberi-AVL, esse però richiedono un enorme
quantitativo di memoria extra, dovuto alla necessità di creare una copia di tutto il sotto albero del nodo che si
intende modificare, con il rischio che al termine di questa operazione, il nodo non sia più valido perchè un altro
Writer ha alterato un altro settore. Non si è quindi ritenuto opportuno procedere per una simile strada, e si è
favorita la strada di uno \underline{Spinlock RW}.

\newpage

\section{Char Device} \label{CharDevice}

Al fine di poter osservare lo stato del sistema, è stato prevista l'aggiunta di un device-driver, che permetta di
ottenere una stringa inerente lo stato del sistema.
Per implementare il driver sono state usate le seguenti \texttt{File Operation Function}:

\begin{description}
 \item [open] :\\
Viene fatto uno ''Screenshot`` dello stato attuale del sistema sotto forma di stringa, e l'istantanea viene salvata
nel \textbf{Per-FileDescriptor Memory}.
 \item [read] :\\
 Permette di trasferire la stringa al lettore contenente le informazioni del sistema.
 \item [lseek] :\\
 Permette di muovere lo spiazzamento di lettura per poter tornare indietro, in caso di bisogno
 \item [relase] (close) :\\
 Permette di cancellare la memoria precedentemente allocata per bufferizzare lo ''Screenshot`` del sistema
\end{description}


Usando queste 4 operazioni è possibile far leggere lo stato del sistema al momento della Open e aggiornando, chiudendo
e riaprendo il File Descriptor.

\subsubsection{Analisi dell'Output} \label{OutputAnalisi}

Essendo il sistema organizzato con 2 alberi, la soluzione migliore per mostrare lo stato, tanto del sistema, quanto
degli alberi, è stata usare ogni riga per rappresentate un \textbf{Nodo}, e le righe \textbf{Superiori} e
\textbf{Inferiori} per i propri figli.

Per leggere lo stato della stanza abbiamo che le informazioni sono:
\begin{enumerate}
 \item \tagSys e \keySys della stanza, e il suo indirizzo di memoria.
 \item Pid del creatore e tipo della stanza \textit{Open for all=0}/\textit{Usabile solo dal processo creante=1}.
 \item Lista dei \Topic in cui vi è almeno 1 \Reader in attesa, se non è presente nessun Reader viene scritto ''no
Waiters``
 \item Somma dei \Reader in attesa in tutta la stanza
\end{enumerate}

I numeri tra parentesi rappresentano il fattore di sbilanciamento dell'Albero-AVL e per via della struttura non possono
mai essere diversi da: -1, 0, +1.

Nel \tagTree sono presenti effettivamente tutte le stanze del sistema; il \keyTree raccoglie solo quelle che hanno
visibilità pubblica.

\subsubsection{Example of Output}

Di seguito l'output del sistema dopo aver eseguito il 7° test, che apre, parla e comunica a caso con un migliaio di
processi, alcuni topic non vengono quindi mai scritti, e si ha l'effetto che rimangono in attesa.

In seguito, digitando \verb|Ctrl+C| viene mandato a tutti i Thread un sigint che non essendo gestito user-space, porta
al risveglio dalla wait, ma causa anche la terminazione del processo, ciò lascia un sistema ''Sporco`` e senza lettori,
mostrato nel secondo listato:

\newpage

\begin{center}\begin{large}
System Screen-shot after running \verb |./test/7_roomExange_LOAD.out|
\end{large}\end{center}
\begin{indentPar}{1.25cm}
\begin{scriptsize}\begin{verbatim}
tagTree (#room = 6; maxRoom = 256):
     .--{@tag(61)-@key(3)-->000000007ce9f0c4} [Creator=13919-perm=0 |L1=77| sum = 77]
 (-1)
    |     `--{@tag(51)-@key(7)-->000000008aec9ce4} [Creator=13459-perm=0 |L1=113| sum = 113]
 (+0)
---{@tag(45)-@key(5)-->000000002e6a0bb1} [Creator=13185-perm=0 |L1=104| sum = 104]
 (+0)
    |     .--{@tag(23)-@key(6)-->000000007894bbe5} [Creator=11755-perm=0 |L1=112| sum = 112]
 (+0)
     `--{@tag(22)-@key(0)-->00000000b0a6d2fa} [Creator=11753-perm=0 |L1=93| sum = 93]
 (+0)
          `--{@tag(10)-@key(2)-->00000000a6a5c13b} [Creator=11209-perm=0 |L1=92| sum = 92]
 (+0)


keyTree:
     .--{@tag(51)-@key(7)-->000000008aec9ce4} [Creator=13459-perm=0 |L1=113| sum = 113]
 (-1)
    |     `--{@tag(23)-@key(6)-->000000007894bbe5} [Creator=11755-perm=0 |L1=112| sum = 112]
 (+0)
---{@tag(45)-@key(5)-->000000002e6a0bb1} [Creator=13185-perm=0 |L1=104| sum = 104]
 (+0)
    |     .--{@tag(61)-@key(3)-->000000007ce9f0c4} [Creator=13919-perm=0 |L1=77| sum = 77]
 (+0)
     `--{@tag(10)-@key(2)-->00000000a6a5c13b} [Creator=11209-perm=0 |L1=92| sum = 92]
 (+0)
          `--{@tag(22)-@key(0)-->00000000b0a6d2fa} [Creator=11753-perm=0 |L1=93| sum = 93]
 (+0)
\end{verbatim}\end{scriptsize}
\end{indentPar}

\begin{center}\begin{large}
System Screen-shot after sending the signal \verb|Ctrl+C|
\end{large}\end{center}

\begin{indentPar}{1.5cm}
\begin{scriptsize}\begin{verbatim}
tagTree (#room = 6; maxRoom = 256):
     .--{@tag(61)-@key(3)-->000000007ce9f0c4} [Creator=13919-perm=0 <no Waiters>]
 (-1)
    |     `--{@tag(51)-@key(7)-->000000008aec9ce4} [Creator=13459-perm=0 <no Waiters>]
 (+0)
---{@tag(45)-@key(5)-->000000002e6a0bb1} [Creator=13185-perm=0 <no Waiters>]
 (+0)
    |     .--{@tag(23)-@key(6)-->000000007894bbe5} [Creator=11755-perm=0 <no Waiters>]
 (+0)
     `--{@tag(22)-@key(0)-->00000000b0a6d2fa} [Creator=11753-perm=0 <no Waiters>]
 (+0)
          `--{@tag(10)-@key(2)-->00000000a6a5c13b} [Creator=11209-perm=0 <no Waiters>]
 (+0)


keyTree:
     .--{@tag(51)-@key(7)-->000000008aec9ce4} [Creator=13459-perm=0 <no Waiters>]
 (-1)
    |     `--{@tag(23)-@key(6)-->000000007894bbe5} [Creator=11755-perm=0 <no Waiters>]
 (+0)
---{@tag(45)-@key(5)-->000000002e6a0bb1} [Creator=13185-perm=0 <no Waiters>]
 (+0)
    |     .--{@tag(61)-@key(3)-->000000007ce9f0c4} [Creator=13919-perm=0 <no Waiters>]
 (+0)
     `--{@tag(10)-@key(2)-->00000000a6a5c13b} [Creator=11209-perm=0 <no Waiters>]
 (+0)
          `--{@tag(22)-@key(0)-->00000000b0a6d2fa} [Creator=11753-perm=0 <no Waiters>]
 (+0)
\end{verbatim}\end{scriptsize}
\end{indentPar}

