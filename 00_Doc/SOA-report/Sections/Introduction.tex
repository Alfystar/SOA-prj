\section{Introduzione}

Il progetto ha l'obiettivo di mettere in comunicazione un numero imprecisato di processi all'interno dello stesso
kernel, e permettere lo scambio di un messaggio tra 1 \Writer e gli N-\Reader precedentemente in attesa su di un canale.

La comunicazione viene svolta nel minor tempo possibile e adottando un algoritmo \textit{Semi-Lock Free} (permette e
favorisce la concorrenza nelle operazioni di scambio di messaggi, ma blocca nelle operazione di creazione e rimozione
di stanze dal sistema).

Il comportamento desiderato è che l'arrivo di 1-\Writer per scrive su di un canale, porti al risveglio degli N-\Reader
precedentemente in attesa sullo stesso, il tutto nel minor tempo possibile.

\subsection{Identificazione dei canali di comunicazione}
Il canale di comunicazione è definito su 2 livelli di ricerca:
\begin{enumerate}
    \item \textbf{Tag-level search} \\
    Stanze effettivamente in-stanziate nel sistema, di default ne possono esistere fino a 256,
    ma è possibile variare a \RunTime questo limite facendolo crescere a piacere, o decrescere fino al
    numero di stanze attualmente aperto, e comunque non meno di 256 \textbf{nell'implementazione di questo progetto}
    (vedi la Sezione ``\nameref{specialFeature}'').
    \item \textbf{Topic-level} \\
    Ogni stanza, possiede a sua volta 32 sotto livelli dove effettivamente i \Reader e il \Writer  possono parlare.
\end{enumerate}

Segue che nella stessa stanza, identificata per mezzo di un \tagSys, è possibile svolgere in parallelo più
conversazioni, e far accodare i diversi Thread sui diversi \Topic, fino al limite di 32 \Topic per
stanza.

Le stanze possono essere pubbliche, e in tal caso gli si associa una \keySys, o private, e in tal caso la \tagSys è
nota solo al creatore della stanza.

\subsection{System behavior}

Per interagire con il sistema sono state implementate 4 system-call "rubate" alla sys-call table dalle funzioni che
puntavano alla \verb|sys_ni_syscall|, ovvero quelle ancora implementate nel kernel.

\begin{enumerate}
    \item \verb|int tag_get(int key, int command, int permission)|\\
    Per me mezzo di questa syscall è possibile creare una stanza con visibilità pubblica o privata (key) per i
    Thread degli altri processi, e decidere se vi possono lavorare o meno (permission)
    \item \verb|int tag_send(int tag, int level, char* buffer, size_t size)|\\
    Questa syscall permette a un \Writer di pubblicare nella stanza desiderata al \Topic voluto
    \item \verb|int tag_receive(int tag, int level, char* buffer, size_t size)|\\
    Questa syscall mette un \Reader in attesa nella stanza al \Topic desiderato finchè un \Writer non la risveglia
    \item \verb|int tag_ctl(int tag, int command)|\\
    Con quest'ultima syscall è possibile inviare dei comandi all'intera stanza, nella fattispecie la chiusura della
    stessa (possibile solo se nessun \Reader è in attesa dentro la Stanza) o il wake-up forzato di tutti i \Reader
    nella stanza.
\end{enumerate}
Per avere più dettagli sulle richieste delle interfacce si faccia riferimento alla pagina ufficiale del progetto:
\href{https://francescoquaglia.github.io/TEACHING/AOS/PROJECTS/project-specification-2020-2021.html}{SOA-prg}

\subsection{Special Features} \label{specialFeature}
\textbf{In aggiunta alle richieste del progetto}, questa implementazione ha la possibilità di modificare a
\RunTime il numero di di Stanze massime del sistema, facendolo crescere a piacere, e ridurre fino al numero minimo di
256, senza però scendere sotto il numero di stanze correntemente allocate.


Viene anche fornita la libreria \texttt{user-space} che permette di trovare autonomamente i numeri delle syscall,
esposti dentro un \verb|module_param_array| al path:
\begin{footnotesize}\verb|/sys/module/TAG_DataExchange/parameters/sysCallNum|\end{footnotesize}, e in aggiunta è
implemento il
\texttt{perror} standard per ciascuna delle 4 syscall.

Per terminare, sono forniti diversi codici di test e gli script che permettono di interagire attraverso il terminale
con il modulo.

\subsection{SubSystem division}

Il progetto è organizzato in 3 diversi sotto sistemi:
\begin{enumerate}
    \item \nameref{SysCallDiscovery} \\
    Sviluppato direttamente dal repository del professore
    (\href{https://github.com/FrancescoQuaglia/Linux-sys_call_table-discoverer}{Git repository}), il quale è stato
    modificato per farlo diventare una libreria integrata nel sistema, con qualche funzione di interfaccia
    \item \nameref{tbde} \\
    Libreria core del sistema, essa implementale funzionalità richieste per il progetto
    \item \nameref{CharDevice} \\
    Libreria che permette di esporre il Modulo al resto del sistema operativo facendolo passare per un dispositivo a
    caratteri
\end{enumerate}

I Sotto-sistemi sono organizzati affinché lavorino con il maggior livello di interdipendenza possibile, facendoli
mettere in
comunicazione dentro \textit{main.c}.
